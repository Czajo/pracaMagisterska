\documentclass[wmii,inf,mgr]{uwmthesis}

\usepackage[MeX]{polski}
\usepackage[utf8]{inputenc}
\usepackage{url}

\date{2017}
\title{Aplikacja okienkowa sterująca systemem Jeden z Dziesięciu}
\author{Maciej Czajkowski}
\etitle{Window application to control One of Ten System}
\wykonanaw{katedrze Metod Matematycznych Informatyki}
\ewykonanaw{XXX}

\podkierunkiem{dr Krzysztofa Sopyły}
\epodkierunkiem{PhD Christopher Sopyła}

\begin{document}
	
\maketitle
	
\tableofcontents

\chapter*{Wstęp}
Lorem ipsum dolor sit amet, consectetur adipiscing elit. Phasellus nec ligula auctor, consectetur arcu id, rutrum tellus. Donec laoreet elementum vulputate. Curabitur congue rutrum justo nec ultrices. Ut orci lacus, dignissim sed facilisis non, dictum eu lectus. Suspendisse potenti. Nulla commodo et ex eu commodo. Integer sit amet leo tincidunt, pharetra sem sed, interdum diam. 

Pellentesque rutrum mi quis ante ultricies laoreet. In consectetur lobortis sem, et vestibulum diam euismod in. Vestibulum facilisis et sapien in dapibus. Etiam ac molestie neque. Aliquam non erat cursus, malesuada tortor sit amet, porttitor dui. Cras iaculis neque at metus viverra, vel rhoncus arcu sollicitudin. Aenean nec tellus pellentesque, tempor orci congue, posuere metus. Ut a tellus tincidunt, sagittis risus quis, suscipit diam. Nunc risus dolor, eleifend eu pharetra ac, viverra suscipit ipsum.

\chapter{Opis problemu}
Curabitur congue rutrum justo nec ultrices. Ut orci lacus, dignissim sed facilisis non, dictum eu lectus. Suspendisse potenti. Nulla commodo et ex eu commodo. Integer sit amet leo tincidunt, pharetra sem sed, interdum diam.
\section{Proponowane rozwiązanie}
\section{Sekcja 2}
\section{Sekcja 3}

\chapter{Budowa stanowisk}
(użyte materiały etc)
\section{Połączenia}
\section{Przekaźniki}
\section{Sekcja 3}

\chapter{Implementacja}
Aliquam non erat cursus, malesuada tortor sit amet, porttitor dui. Cras iaculis neque at metus viverra, vel rhoncus arcu sollicitudin.
\section{Sprzęt (Raspberry Pi)}
\section{Sekcja 2}
\section{Sekcja 3}

\chapter*{Bibliografia}

\end{document}